\usepackage[utf8]{inputenc}
\section{\documentclass[12pt,a4paper]{article}
\usepackage{graphicx}
\begin{titlepage}
\newcommand{\HRule}{\rule{\linewidth}{0.5mm}} % Defines a new command for the horizontal lines, change thickness here

%----------------------------------------------------------------------------------------
%	LOGO SECTION
%----------------------------------------------------------------------------------------
\centering
\includegraphics[width=5cm]{logo nsu.jpeg}\\
%\includegraphics[width=5.5cm]{title/logo nsu.jpeg}\\[1cm] % Include a department/university logo - this will require the graphicx package
 
%----------------------------------------------------------------------------------------

\center % Center everything on the page

%----------------------------------------------------------------------------------------
%	HEADING SECTIONS
%----------------------------------------------------------------------------------------
\textsc{\LARGE \textbf{\textit{North South University}}\\[1cm]\textbf{\textit{Department of Electrical and Computer Engineering}}}\\[0.5cm]
\textsc{\LARGE \textbf{Project Proposal}}\\[2cm] 


%----------------------------------------------------------------------------------------
%	TITLE SECTION
%----------------------------------------------------------------------------------------
\makeatletter
\HRule \\[0.4cm]
{ \huge \bfseries\textbf{ ❝Hospital  Management  System❞}}\\[0.4cm] % Title of your document
\HRule \\[0.5cm]


 
%----------------------------------------------------------------------------------------
%	AUTHOR SECTION
%----------------------------------------------------------------------------------------

\begin{minipage}{0.4\textwidth}
\begin{flushleft} \large
\emph{Students Name and Id:}\\

\textbf{Md.Asaduzzaman Chowdhury 2021355642}

\textbf{Md.Masud Shohail 2022123642}

\textbf{Shahran Rahman Alve 2022253642}% Your name



\end{flushleft}
\end{minipage}
~
\begin{minipage}{0.4\textwidth}
\begin{flushright} \large
\emph{Faculty:}\\
\textbf{\textbf{Nadeem Ahmed}}\\[1.5cm]
\emph{Submitted to:} \\
\textbf{Nazmul Alam Diptu} \\[1.2em] 
\end{flushright}
\end{minipage}\\[2cm]
\makeatother

%----------------------------------------------------------------------------------------
%	DATE SECTION
%----------------------------------------------------------------------------------------

{\large February 19,2022}\\[2cm] % Date, change the \today to a set date if you want to be precise

\vfill %


\end{titlepage}

\begin{document}


\section{Introduction}
The project "Hospital Management System" aims to computerize the Office 
Management of the Hospital to develop user-friendly, simple, fast, and cost-effective 
software. It is concerned with gathering patient information, diagnosis data, and so on. 
Previously, it was done by hand.The existing method requires plenty of paper forms, with data repositories spread over the hospital's administrative system. Information on forms is frequently missing or fails to reach management standards. Documents are frequently lost in transit between departments, requiring a thorough auditing process to guarantee that no essential data is lost. In the hospital, there are many copies of the same information, which might lead to data inconsistencies across different data storage. Patient's personal information, medical history, hospital employee information, ward and schedule, operating theater scheduling, and various facility waiting lists are often included in this information. All of this data needs to be managed in a reliable, error-free, and cost-effective manner. Our management system intends to include organizing, consolidating, maintaining data integrity, and preventing discrepancies.
Additionally,  .The system's main job is to register and maintain 
patient and doctor information and access and meaningfully change this information 
as needed. Patient information and diagnosis information are entered into the system, 
and the system output is used to display these details on the screen. A username and 
password are required to access the Hospital Management System. A receptionist or 
an administrator can access it. They are the only ones who have access to the database. 
The information is easily accessible. The data is well-protected for personal use and 
the data processing is rapid

\section{Objective}
\begin{itemize}
\item To computerize all details regarding patient details & hospital details.
\item The information of the patient should be kept up to date and there record should be kept in the system for historial purposes.
\item keep in track of the all employees.
\item Help to provide the test reports of patients conducted by the pathology.
\end{itemize}

\section{Web Application Feature and description}
\\The web page will open with the customer view at first. It will be general view
and anyone accessing the domain will be able to view it.Viewer  has to create an account first for service. Once a viwer has created an account, he/she can:
\begin{itemize}
\item Login
\item Registration
\item Patient Module
\item Pathology
\item  Profile Update
\item Update/Delete List
\item Payment Method
\item Billing Method
\end{itemize}

\section{Table Names In Database(Based on Assumption)}
\begin{enumerate}
\item Admin
\item Patient Details 
\item Doctor Details 
\item Room Details
\item staff Details
\item Billing
\end{enumerate}



\newpage
\section{Tools and Resources(Based On Assumption)}
\begin{itemize}
\item HTML
\item CSS
\item MySQL
\item PHP
\item Java Script
\item Web Server
\item Google Maps (API)
\item SMS (API)

\section{Challenge}
The major concern is the significant cost of software design and implementation. Because of the combination of existing manual and digital procedures, the Hospital Management System may encounter various obstacles. On the other hand, the database's size increases by the day; expanding the database's capacity and backup and data maintenance activities provides several challenges. Employees who will be working with the system need to be trained in basic computer functions. Ultimately, the fear of data security breaches and the complicated design in terms of the User Interface and Experience are also the facts that must constantly be kept in mind.




\end{document}
